\documentclass[tikz,crop,convert={density=400,outext=.png}]{standalone}

\usepackage{booktabs}
\usepackage{etoolbox}
\usepackage[referable]{threeparttablex}
\usepackage[round-mode=places,detect-weight=true,detect-inline-weight=math]{siunitx}
\renewcommand\arraystretch{1.2}

\listfiles

\begin{document}
\begin{table}
%\centering\tlstyle
\begin{threeparttable}
\robustify\bfseries
%\caption{Mean and standard deviation whole-brain normalized SUV values from [$^{18}$F]FDG-PET and group statistics.}
\input{stats_table.tbl}
    \begin{tablenotes}
      \item\footnotesize{\textbf{Abbreviations:} Veh =  vehicle, DMT = \textit{N,N}-dimethyltryptamine, Har = harmine, 3-IAA = 3-indole acetic acid, Front Cort = frontal cortex. \\
For the compounds DMT, 3-IAA and harmine the concentrations are given for each group in  frontal cortex and cerebellum. Values in columns 3-6 represent mean (SD) concentrations in $\mu$g/L.  \\N = 6 for each group.}
    \end{tablenotes}
  \end{threeparttable}
\end{table}
%\footnotesize{\textbf{Abbreviations:} mPFC, medial prefrontal cortex; OFC, orbitofrontal cortex; NAc, nucleus accumbens; Har, harmine, Veh, vehicle; df, degrees of freedom. \\
%Values in columns 2-5 represent mean (SD) in SUV values per group, N = 6 for \textit{Har + DMT, DMT, Har} and N = 5 for \textit{Veh}.  ANOVA was done for df = degrees of freedom. n~=~9.}
\end{document}